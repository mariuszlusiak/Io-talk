\begin{frame}
  \begin{center}
    The features of the Io language
  \end{center}
  \note{
    \begin{itemize}
      \item many approaches to new language
      \item hello world tutorial
      \item sit down and hack away
      \item interactive console
      \item as professionals we should make an scientific approach
      \item ask general questions, good questions, that lead to meaningful answers
      \item that's what makes us different from high school kids
      \item what kind of questions can we ask?
    \end{itemize}
  }
\end{frame}

\begin{frame}
  \begin{center}
    Interpreted or compiled or both?
    \vskip15pt
    \onslide<2->
    Interpreted.
  \end{center}
  \note{
    \begin{itemize}
      \item determines interaction programmer - language
    \end{itemize}
  }
\end{frame}

\begin{frame}
  \begin{center}
    Programming paradigm?
    \vskip15pt
    \onslide<2->
    
    Object oriented.
  \end{center}
  \note{
    \begin{itemize}
      \item structural
      \item object-oriented
      \item functional
    \end{itemize}
  }
\end{frame}

\begin{frame}
  \begin{center}
    Prototype-based
  \end{center}
  \note{
    \begin{itemize}
      \item in Ruby we work with objects all the time
      \item but Io has no classes!
    \end{itemize}
  }
\end{frame}

\begin{frame}
  \begin{center}
    No classes!
  \end{center}
  \note{
    \begin{itemize}
      \item you might be wondering
      \item how objects are created?
      \item in Ruby object is an instance of a class
      \item how is inheritance implemented
      \item we'll get deeper in a moment
    \end{itemize}
  }
\end{frame}

\begin{frame}
  \begin{center}
    Syntax?
    \vskip15pt
    \onslide<2->
    
    Simple.
  \end{center}
  \note{
    \begin{itemize}
      \item syntax is next important thing
      \item how long will it take to learn a language?
      \item Io has simple syntax
    \end{itemize}
  }
\end{frame}

\begin{frame}
  \begin{center}
    Expressive?
    \vskip15pt
    \onslide<2->
    
    Not really.
  \end{center}
  \note{
    \begin{itemize}
      \item little syntactic sugar
      \item in Ruby you can express complex thoughts in little writing
    \end{itemize}
  }
\end{frame}

\begin{frame}[fragile]
  \begin{center}
    In Ruby:
    \vskip15pt
    \begin{lstlisting}
      arr = [1, 2, 3]
      arr[-1] # => 3
    \end{lstlisting}
  \end{center}
  \note{
    \begin{itemize}
      \item this is a nice syntax if you know it
    \end{itemize}
  }
\end{frame}

\begin{frame}
  \begin{center}
    Simple syntax.
  \end{center}
  \note{
    \begin{itemize}
      \item makes it easy to write things 
      \item makes it harder to understand complex thoughts
    \end{itemize}
  }
\end{frame}

\begin{frame}
  \begin{center}
    Clean syntax.
  \end{center}
\end{frame}

\begin{frame}
  \begin{center}
    Comparing to Perl...
  \end{center}
  \note{
    \begin{itemize}
      \item write-only language
    \end{itemize}
  }
\end{frame}

\begin{frame}
  \begin{center}
    Clean syntax.
  \end{center}
  \note{
    \begin{itemize}
      \item little strange chars
    \end{itemize}
  }
\end{frame}

\begin{frame}
  \begin{center}
    Type system
  \end{center}
\end{frame}

\begin{frame}
  \begin{center}
    Weak or strong typing?
    \vskip15pt
    \onslide<2->
    
    Strong.
  \end{center}
  \note{
    \begin{itemize}
      \item var = 1, var = aaa
      \item 3 plus string in Ruby, in Javascript
      \item conditional statements (compare to Java)
    \end{itemize}
  }
\end{frame}

\begin{frame}
  \begin{center}
    Static or dynamic typing?
    \vskip15pt
    \onslide<2->
    
    Dynamic.
  \end{center}
\end{frame}

\begin{frame}
  \begin{center}
    Support for concurrency?
    \vskip15pt
    \onslide<2->
    
    Strong.
  \end{center}
\end{frame}

\begin{frame}
  \begin{center}
    Not language features but important.
  \end{center}
\end{frame}

\begin{frame}
  \begin{center}
    Community?
    \vskip15pt
    \onslide<2->
    
    Small.
  \end{center}
\end{frame}

\begin{frame}
  \begin{center}
    Testing frameworks?
    \vskip15pt
    \onslide<2->
    
    UnitTest.
  \end{center}
\end{frame}
